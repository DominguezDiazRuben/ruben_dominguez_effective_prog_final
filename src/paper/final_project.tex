\documentclass[11pt, a4paper, leqno]{article}
\usepackage{a4wide}
\usepackage[T1]{fontenc}
\usepackage[utf8]{inputenc}
\usepackage{float, afterpage, rotating, graphicx}
\usepackage{epstopdf}
\usepackage{longtable, booktabs, tabularx}
\usepackage{fancyvrb, moreverb, relsize}
\usepackage{eurosym, calc}
% \usepackage{chngcntr}
\usepackage{amsmath, amssymb, amsfonts, amsthm, bm}
\usepackage{caption}
\usepackage{mdwlist}
\usepackage{xfrac}
\usepackage{setspace}
\usepackage{xcolor}
\usepackage{subcaption}
\usepackage{minibox}
% \usepackage{pdf14} % Enable for Manuscriptcentral -- can't handle pdf 1.5
% \usepackage{endfloat} % Enable to move tables / figures to the end. Useful for some submissions.



\usepackage{natbib}
\bibliographystyle{rusnat}




\usepackage[unicode=true]{hyperref}
\hypersetup{
    colorlinks=true,
    linkcolor=black,
    anchorcolor=black,
    citecolor=black,
    filecolor=black,
    menucolor=black,
    runcolor=black,
    urlcolor=black
}


\widowpenalty=10000
\clubpenalty=10000

\setlength{\parskip}{1ex}
\setlength{\parindent}{0ex}
\setstretch{1.5}


\begin{document}

\title{Effective Programming Final Project\thanks{Ruben Dominguez Diaz, Bonn University. Email: \href{mailto:s6rudomi@uni-bonn.de}{\nolinkurl{s6rudomi [at] uni-bonn [dot] de}}.The solution is based on the template for reproducible research projects by \citet{GaudeckerEconProjectTemplates}. This research project build on an earlier version of a paper that I handed in for the course "Topic Course in Macroeconomics and Public Economics: Household Portfolios in Macroeconomics.". The main value added of this project is polished code.} }

\author{Ruben Dominguez Diaz}

\date{
{\bf Matrikel Nr 3014599} 
\\[1ex] 
\today
}

\maketitle

\begin{abstract}
	This research project uses the Survey of Consumer Finances 2016 to present some summary stastitics about income and wealth inequality. In addition, it reports the average income, wealth, age and balance sheet composition of households along the income and wealth distribution, as well as over the business cycle.
\end{abstract}
\clearpage

\section{Introduction} 
\label{sec:introduction}
    Households differ along different dimensions, such age, income, wealth and asset portfolios. At the same time, an increasing literature in macroeconomics has emphasized the relevance of inequality to answer important research questions. Yet, the foundation of those analysis crucially depend on our understanding on how heterogeneiy looks like in the data. Therefore, a key step in understanding the role of inequality in macroeconomics is a detailed and granual analysis of the data that we have available.

    In order to make progress in this direction, this paper uses the Survey of Consumer Finances (SFC) to provide an snapshot of the US economy in 2016 at the household level. In particular, I closely follow the methodology employed in \citet{Kuhn2016}. To start, I provide broad measures of inequality that are commonly used in the literature. More precisely, I compute the Gini coefficients for income and wealth in the sample, and show that, as expected, wealth is substantially more concentrated than income. To further emphasize this point graphically, I provide the estimated Lorenz curves and histograms for both variables. Both measures show the high skewness of the income and wealth distribution, in particular of the latter.

    In a second step, I look closer to the characteristics of the households. In particular, I partition the sample along the wealth and income distribution, as well as according to the the life cycle. After subdividing the sample into age or wealth and income bins, I characterize the representative household in each group. More precisely, I provide the average wealth, income, age and main portfolio composition of the representative household in each bin. This turns out to be useful because, for example, the income rich households are not the same as the wealth rich households and hence differ along several characteristics. 

    This paper follows and builds on a large series of papers analyzing the Survey of Consumer finances (e.g. \citet{Diaz-Gimenez1997}, \citet{Diaz-Gimenez2011}), the most recent of which is \citet{Kuhn2016}, who analyzes the SFC for the 2013 wave. As emphasized earlier, the structure of the analysis largely is inspired by these previous papers. In addition, following \citet{Hintermaier2016} I also report shares of secured debd.

    The paper paper is organized as follows. Section 2 reports the main results and Section 3 concludes.


\section{Main Results.}
\label{sec:mainresults}

This section presents the main results of the research project.
\begin{figure}
    \caption{Lorenz Curve for Income}
    
    \includegraphics[width=\textwidth]{../../out/figures/lorenz_income}
    \label{fig:lorincome}
\end{figure}


\begin{figure}
    \caption{Lorenz Curve for Net Worth}
    
    \includegraphics[width=\textwidth]{../../out/figures/lorenz_networth}
    \label{fig:lornetworth}

\end{figure}

\begin{figure}
    \caption{Histogram for Income}
    
    \includegraphics[width=\textwidth]{../../out/figures/histogram_income}
    \label{fig:histincome}
\end{figure}

\begin{figure}
    \caption{Histogram for Net Worth}
    
    \includegraphics[width=\textwidth]{../../out/figures/histogram_networth}
    \label{fig:histnetworth}
\end{figure}

\begin{table}
    \caption{Quantiles of the Income distribution.}
    \resizebox{\textwidth}{!}{\input{../../out/tables/income_quintiles_table}}
    \label{tab:quintilesincome}
\end{table}


\begin{table}
    \caption{Quantiles of the Wealth distribution.}
    \resizebox{\textwidth}{!}{\input{../../out/tables/net_worth_quintiles_table}}
    \label{tab:quintilesnetworth}
\end{table}

\begin{table}
    \caption{Age partition.}
    \resizebox{\textwidth}{!}{\input{../../out/tables/age_partition}}
    \label{tab:agepartition}
\end{table}

\section{Conclusions.}






\bibliography{refs}


\begin{table}
    \caption{Deciles of the Income distribution.}
    \resizebox{\textwidth}{!}{\input{../../out/tables/income_deciles_table}}
\end{table}

\begin{table}
    \caption{Deciles of the Wealth distribution.}
    \resizebox{\textwidth}{!}{\input{../../out/tables/net_worth_deciles_table}}
\end{table}

% The chngctr package is needed for the following lines.
% \counterwithin{table}{section}
% \counterwithin{figure}{section}

\end{document}
